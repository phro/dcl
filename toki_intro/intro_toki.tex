\documentclass[10.5pt]{article}
\usepackage{amsmath, fancyhdr}
\usepackage{amssymb, amsthm}
\usepackage{bbm}
\usepackage[mathscr]{euscript}
%%%
% Set up the margins to use a fairly large area of the page
%%%
\oddsidemargin=.2in
\evensidemargin=.2in
\textwidth=6in
\topmargin=0in
\textheight=9.0in
\parskip=.07in
\parindent=0in

%%%
% Set up the header
%%%
\newcommand{\setheader}[5]{
    \lhead{{ #1}\\{ #2}}
    \rhead{{ #3}\\{ #4}}
    \bigskip
    \center{{\bf #5}}\\
    \medskip
    %\today
}

\begin{document}
\thispagestyle{fancy}
\setheader{Mathematical and Computational Linguistics}{Dr.  Marcolli}
{University of Toronto}{Jesse Frohlich and Andrew Wilson}{Toki Taster}
\medskip

\begin{quote}
Toki Pona is a human constructed language created by Sonja Lang in $2010$. 
It was an attempt to simplify ideas to their most basic elements. For example,
what is a professor but a `person of knowledge'? And if you're hungry, then
you `want eat'. In Sonja's words,
\begin{quote}
``If English is a thick novel, then Toki Pona is haiku.''
\end{quote}

Enough with the introduction though, let's build some sentences using the sentence
formula ` Noun + li + Noun'. For example, `lipu li ijo' translates to ` a 
book is something'.


\begin{tabular}{c c c c}
{\bf Nouns}  &                          & {\bf Adjectives} &  \\
\emph{ijo}   & something, thing, being      \\ 
\emph{jan}   & person, human                \\ 
\emph{kili}  & fruit, vegetable             \\ 
\emph{nanpa} & number                              \\ 
\emph{lipu}  & document, book                   \\
\emph{telo}  & water                      \\
\emph{sona}  & knowledge                  \\
\emph{toki}  & language                     \\
\end{tabular}

\end{quote}
\end{document}
