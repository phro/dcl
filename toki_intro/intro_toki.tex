\documentclass[10.5pt]{article}
\usepackage{amsmath, fancyhdr}
%%%
% Set up the margins to use a fairly large area of the page
%%%
\oddsidemargin=.2in
\evensidemargin=.2in
\textwidth=6in
\topmargin=0in
\textheight=9.0in
\parskip=.07in
\parindent=0in

%%%
% Set up the header
%%%
\newcommand{\setheader}[5]{
    \lhead{{ #1}\\{ #2}}
    \rhead{{ #3}\\{ #4}}
    \bigskip
    \center{{\bf #5}}\\
    \medskip
    %\today
}

\begin{document}
\thispagestyle{fancy}
\setheader{Mathematical and Computational Linguistics}{Dr.  Marcolli}
{University of Toronto}{Jesse Frohlich and Andrew Wilson}{Toki Taster}
\medskip

\begin{quote}
Toki Pona is a human constructed language created by Sonja Lang in $2010$. 
It was an attempt to simplify ideas to their most basic elements. For example,
what is a professor but a `person of knowledge'? And if you're hungry, then
you `want eat'. In Sonja's words,
\begin{quote}
``If English is a thick novel, then Toki Pona is a haiku.''
\end{quote}

Enough with the introduction though, let's build some sentences using the sentence
formula ` Noun + li + Noun'. For example, `lipu li ijo' translates to ` a book is something'.
If you'd like to add adjectives, they come after nouns, I.e, `suli jan li wawa` translates

\medskip

\begin{tabular}{c c c c}
{\bf Nouns}  &                          & {\bf Adjectives} &                              \\
\emph{ijo}   & something, thing, being  & \emph{suli}      & big, large                   \\ 
\emph{jan}   & person, human            & \emph{lili}      & little, small                \\ 
\emph{mi}    & I, me, we                & \emph{pona}      & good, simple, friendly       \\ 
\emph{sina}  & you                      & \emph{wawa}      & strong, confident, powerful  \\ 
\emph{ni}    & this, that               & \emph{mute}      & many, very, a lot            \\ 
\emph{nanpa} & number                   & \emph{ala}       & no, not, zero                \\ 
\emph{lipu}  & document, book           & \emph{ike}       & bad, negative                \\
\emph{telo}  & water                    & \emph{sin}       & new, another, fresh          \\
\emph{sona}  & knowledge                &                  &                              \\
\emph{toki}  & language                 &                  &                              \\
\emph{kili}  & fruit, vegetable         &                  &                              \\
\end{tabular}
\medskip

Now let's add simple noun phrases and verbs. Noun phrases can be constructed using 
the formula `Noun Noun', E.g. `lipu sona' translates to `book of knowledge'. 
To build transitive verb phrases, use the formula `Noun + li + Verb + e + Object`.
For example, `jan pona li jo e lipu sona`.

\medskip
\begin{tabular}{c c}
{\bf Verbs}   &                               \\
\emph{jo}     & to have, carry, contain       \\ 
\emph{kute}   & to listen to, obey            \\
\emph{moku}   & to eat, drink                 \\
\emph{pali}   & to make, do, work on          \\         
\emph{sona}   & to know                       \\
\emph{toki}   & to talk, speak, communicate   \\  
\emph{wile}   & to want, need                 \\  
\end{tabular}
\medskip

You've probably realized how incredibly ambiguous this language can be. For instance, 
my shopping list in Toki Pona would look something like: kili, kili, kili, $\dots$. 

Try converting the following sentences from Toki Pona to English:
\begin{enumerate}
  \item mi moku.
  \item mi wile telo.
  \item sona toki pi sona nanpa.
\end{enumerate}

Now from English to Toki Pona:
\begin{enumerate}
  \item I am eating an eggplant.
  \item Our small community is friendly.
  \item Mathematical linguistics is fun.
\end{enumerate}
\end{quote}
\end{document}
